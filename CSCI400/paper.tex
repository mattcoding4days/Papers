\documentclass[a4paper,man,floatsintext]{apa6}
\usepackage[american]{babel}
\usepackage{fancyhdr}
\usepackage[utf8]{inputenc}
\usepackage{csquotes}
\usepackage[style=apa,sortcites=true,sorting=nyt,backend=biber]{biblatex}
\usepackage[colorlinks,citecolor=blue,urlcolor=blue]{hyperref}
\DeclareLanguageMapping{american}{american-apa}

\pagestyle{fancy}
\lfoot{\href{https://github.com/mattcoding4days/Papers/tree/main/CSCI400}{Use the Source Luke}}
\DeclareBibliographyCategory{cited}
\AtEveryCitekey{\addtocategory{cited}{\thefield{entrykey}}}
\nocite{*}

\addbibresource{main.bib}


\title{Decentralized Finance and the Ethical Problems it Solves}
\shorttitle{Decentralized Finance}
\author{Matt Williams}
\affiliation{Vancouver Island University \\ CSCI 400: Computers and Society \\ Dr.Alena Kottova}

\abstract{The history of traditional centralized finance is wrought with inequality, fancy nomenclature,
exclusionary jargon, and has left the world with an imbalanced void. According to \textcite{jeff_desj_2019} ``Globally,
about 1.7 billion adults remain unbanked, without an account at a financial institution or through a mobile
money provider.'' During a time of financial illiteracy and a growing wealth gap, it is clear that new solutions are needed.
Over the past few years, a flock of \emph{DeFi} (Decentralized Finance) companies has
been aiming to solve modern financial problems with fresh perspectives and blockchain technologies.
However, with \emph{DeFi} still being in its infancy, stakeholders from all sides have been sprinting
towards the cause with arms wide open or pitchforks drawn. With the post WW2 boomer economy coming to an end,
we will look at what \emph{DeFi} is, how it works, the technical and ethical problems it solves,
and the issues that it faces going forward.
}

\begin{document}

\maketitle

\section{The Historical Problem}

``The global financial system has created massive wealth, but its centralized nature means the spoils have gone
to the people who are best connected to the financial centers of the world'' \parencite{card_frank_2019}. We live in a world
where we now have reusable rocket ships, electric cars, the internet, and the ability the connect each person to a grid with a
small pocket-sized electronic device. But in these modern times, the term ``third world'' still exists? While
in our first-world economy, banking seems to be fitting the bill, it is clear that the rest of the world is
falling behind. Not only is the current system exclusionary in nature, but it is also expensive and inefficient. With
intermediaries like banks charging an average 10.53\% to send money internationally \parencite{jeff_desj_2019}, with a wait
time of many days, this service is unacceptable.

Possibly one of the most prevalent problems; is that in a centralized system, a countries government has the power to manipulate
and devalue fiat currencies, which in turn can have devastating effects on markets and the lives of citizens \parencite{jeff_desj_2019}.
This exact problem has been seen on the world stage, with China manipulating their own currency to encourage investment from foreign money
to Venezuela continuously printing money to the point of ``runaway hyperinflation'' \parencite{jeff_desj_2019}.

The final crux of a centralized system can be seen in the most recent financial collapse of 2008 in America, where one centralized
failure can cause the entire centralized system to come crashing down. Worst of all, common citizens have no control over any of this.
For these reasons and more, I believe this system is not operating in the best interest of every human on this planet, it could be done
better, and that's what blockchain-based decentralization can offer.

\section{Stakeholders}

``Decentralized finance is essentially just conventional financial tools built on a blockchain, specifcally Ethereum'' \parencite{bryan_curran_2019}.
Being that the blockchain at the end of the day is just a series of protocols that require a cryptographic proof of work. These
blockchains do not implement censorship, they are transparent, open-source and immutable. However the term ``blockchain'' is vague
and often confuses people, but the focus here is the Ethereum blockchain in particular. Ethereum is the protocal that the majority
of \emph{dAPPS}



\section{Ethical Analysis}



\section{Conclusion}

\printbibliography[category=cited]                                                                                           

\end{document} 
